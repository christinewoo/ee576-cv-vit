\documentclass{article} % For LaTeX2e
\usepackage{nips15submit_e,times}
\usepackage{hyperref}
\usepackage{url}
%\documentstyle[nips14submit_09,times,art10]{article} % For LaTeX 2.09


\title{ViT Pokemon Classifier}


\author{
Christine Wu\\
Department of Electrical and Computer Engineering\\
University of Washington\\
Seattle, WA 98105 \\
\texttt{wuc29@uw.edu} \\
\And
Pei-Hsuan Lin \\
Department of Electrical and Computer Engineering\\
University of Washington\\
Seattle, WA 98105 \\
\texttt{peihslin@uw.edu} \\
\AND
Zhiwei Zhong \\
Department of Electrical and Computer Engineering\\
University of Washington\\
Seattle, WA 98105 \\
\texttt{zhongz22@uw.edu} \\
\\
}

% The \author macro works with any number of authors. There are two commands
% used to separate the names and addresses of multiple authors: \And and \AND.
%
% Using \And between authors leaves it to \LaTeX{} to determine where to break
% the lines. Using \AND forces a linebreak at that point. So, if \LaTeX{}
% puts 3 of 4 authors names on the first line, and the last on the second
% line, try using \AND instead of \And before the third author name.

\newcommand{\fix}{\marginpar{FIX}}
\newcommand{\new}{\marginpar{NEW}}

\nipsfinalcopy % Uncomment for camera-ready version

\begin{document}


\maketitle

\begin{abstract}
In this report, we present a comprehensive study on the application of Vision Transformers 
(ViTs) for the task of Pokemon classification. While convolutional neural networks (CNNs) 
have been widely used, the ViT has recently emerged as a promising alternative due to the 
self-attention mechanisms. In this project, we developed a ViT model from scratch to explore
 the functionality of the ViT. The ViT model captures spatial relationships using 
 self-attention mechanisms, enabling it to learn both local and global features. 
 In addition, we fine-tuned a pre-trained model, which was pre-trained on ImageNet-21k at 
 resolution 224x224, to increase the accuracy of our model. Lastly, we discussed the 
 performance of the model that we developed from scratch and the pre-trained model.
\end{abstract}

\section{Related work-Pei-Hsuan}
The field of computer vision has witnessed significant advancements in recent years, 
particularly in the domain of image classification. Traditional approaches for image 
classification heavily relied on convolutional neural networks (CNNs) such as AlexNet 
and VGGNet, which achieved remarkable performance on various benchmark datasets. However, 
the emergence of Vision Transformers (ViTs) has garnered attention as a promising 
alternative for visual recognition tasks.

The concept of Transformers was initially introduced in the field of natural language 
processing (NLP), where it revolutionized machine translation with the Transformer model. 
Transformer was proposed by Vaswani et al. (2017) and it becomes the state-of-the-art 
method in NLP tasks. Inspired by this success, researchers began exploring the application 
of Transformers in computer vision. Cordonnier et al. (2020) showed that images can express 
any convolutional layer by applying the self-attention layer to them. Furthermore, 
Cordonnier et al. (2020) proved that fully-attentional models can combine local behavior 
and global attention based on input content. Dosovitskiy et al. (2021) pioneered the 
concept of Vision Transformers by adapting the Transformer architecture for image 
classification tasks. They demonstrated the effectiveness of ViTs on large-scale image 
datasets like ImageNet, surpassing the performance of CNN-based models.

Since ViTs have shown promise in general image classification tasks, they may be a good 
tool for classifying Pokemon. This report aims to investigate the application of ViTs for 
Pokemon classification and provide insights into the performance and potential of ViTs in 
this specialized domain.

\section{Technical Description-Ted and Christine, DataLoader-Pei-Hsuan}
\label{gen_inst}

% ALGORITHM
Our goal of this project is to learn the ViT architecture, implement from scratch, and
perform image classification on 150 different Pokemon. Below, we describe the individual 
moduals in our self-implemented ViT.

\subsection{Algorithm by Step}

1. This module takes in a 2D image with shape HWC and flattens the image into 2D patches each with shape $N * (P * P * C)$.
Here $N = {H * W} / {P^2}$, which is the number of patches, and $P$ is the number of pixels for each image patch.

2. We map the flattened patches to a dimension $D$, where $D$ is the constant latent vector size throughout the transformer layers.
These traininable linear projection outputs are the patch embeddings.
We prepend a class token in front of each patch embedding to serve as the image representation that will later be used in the
classification head at the end of the model. Lastly, to retain the positional information of each patch relative to the original image,
a 1D learnable positional embeddings are added.

3. The transformer block, shown below in Fig. 2, takes embedded patches as inputs and passes them a multi-headed self-attention layer (MSA)
followed by a MLP block with two layers and a GELU non-linearity layer. Before the MSA and MLP blocks, a block of Layernorm is added.
Additionally, there are residual connections after every block.

4. The multi-head self-attention module takes in the embedded sequence of tokens and computes the global self-attention.
For each token, it is mapped to $v$, $k$, and $q$ and passed into the scaled dot-product attention head.
The scaled dot product is used to compare the "query" with the "keys" and get scores for the "values."
Each of the scores is in short the relevance between the "query" and each "key".
We reweight the "values" with the obtained scores and take the summation of the reweighted "values".
Lastly, it is concatenated and passed into another linear layer for information flow, which helps to mix the information captured by each head.

5. The classification head is implemented with a MLP consisting of one hidden layer during pre-train.
During the fine-tuning stage, the MLP will consist of one single linear layer.



% Pre-train
\subsection{Pretrained Model}
For Transformers, and other self-attention-based architectures, the training usually consists of pre-training on 
a large dataset, such as a large text corpus for natural language processing tasks, and then fine-tune on a 
task-specific dataset. In our case, we want to perform image classification, therefore we used the pre-trained weights
released on Hugging Face. The pretrained weight we have used is "vit-base-patch16-224". It is a ViT model pre-trained
on ImageNet-21k, which consists of 14 millioin images and 21,843 classes. Each of the image has resolution 224x224.
Through this pre-training, the model learns the representation of images that can be used to extract features for the
fine-tune task. The fine-tune training trains a Pokemon classifier by adding a linear layer on top of the pre-trained
ViT encoder for the 150 classes in the Pokemon dataset.

\subsection{Dataloader} % include dataset information

%The version of the paper submitted for review should have ``Anonymous Author(s)'' as the author of the paper.

For the final version, authors' names are
set in boldface, and each name is centered above the corresponding
address. The lead author's name is to be listed first (left-most), and
the co-authors' names (if different address) are set to follow. If
there is only one co-author, list both author and co-author side by side.

Please pay special attention to the instructions in section \ref{others}
regarding figures, tables, acknowledgments, and references.

\section{Experimental results-Ted and Christine}
\label{headings}
We train our from-scratch ViT model on a GTX3060 GPU with the CIFAR-10 dataset. There are 50,000 RGB training images
with resolution 32x32. The use a batch size of 32, learning rate of 0.0002. The number of hidden dimension is 128,
number of heads is 4, and the number of transformer blocks is 4. We train the model for 30 epochs. Our PokemonClassifier
that utilizes the pre-trained weights on ImageNet-21K was trained on a GTX3060 GPU, with batch size of 16 and learning 
rate of $5 * 10^{-5}$ for 5 epochs. 

We present our experimental results for the from-scratch model and pretrained model, respectively:

% \subsection{Experimental Results for Model from Scratch}

\begin{table}[ht]
  \centering
    \begin{tabular}{ c | c | c | c | c  | c | c }
      \hline
      Image Size & Learning Rates & Hidden Dimension & Heads & Number of Blocks & Epoch & Accuracy \\ \hline
      3 x 32 x 32 & 0.0002 & 128 & 4 & 4 & 30 & 62.64\% \\ \hline
      3 x 32 x 32 & 0.001 & 64 & 2 & 6 & 20 & 57.94\% \\ \hline
    \end{tabular}
    \caption{Best Training Results of the Model from Scratch}
  \label{tab:my_label}
\end{table}


% \subsubsection{Experimental Results for Pretrained Model}
\begin{table}[ht]
  \centering
    \begin{tabular}{ c | c | c | c | c  | c }
      \hline
      Item & Epoch 1 & Epoch 2 & Epoch 3 & Epoch 4 & Epoch 5 \\ \hline
      validation loss & 2.62 & 0.9648 & 0.5622 & 0.4275 & 0.371\\ \hline
      validation accuracy & 0.7287  & 0.9124 & 0.9432 & 0.952 & 0.952 \\ \hline
    \end{tabular}
    \caption{Training Results of the Pretrained Model}
  \label{tab:my_label}
\end{table}

\section{Discussion of results-All}
\label{others}

These instructions apply to everyone, regardless of the formatter being used.

\begin{figure}[h]
\begin{center}
%\framebox[4.0in]{$\;$}
\fbox{\rule[-.5cm]{0cm}{4cm} \rule[-.5cm]{4cm}{0cm}}
\end{center}
\caption{Sample figure caption.}
\end{figure}

\subsection{Tables}

All tables must be centered, neat, clean and legible. Do not use hand-drawn
tables. The table number and title always appear before the table. See
Table~\ref{sample-table}.

Place one line space before the table title, one line space after the table
title, and one line space after the table. The table title must be lower case
(except for first word and proper nouns); tables are numbered consecutively.

\begin{table}[t]
\caption{Sample table title}
\label{sample-table}
\begin{center}
\begin{tabular}{ll}
\multicolumn{1}{c}{\bf PART}  &\multicolumn{1}{c}{\bf DESCRIPTION}
\\ \hline \\
Dendrite         &Input terminal \\
Axon             &Output terminal \\
Soma             &Cell body (contains cell nucleus) \\
\end{tabular}
\end{center}
\end{table}

\section{Future works-Christine}


Do not change any aspects of the formatting parameters in the style files.
In particular, do not modify the width or length of the rectangle the text
should fit into, and do not change font sizes (except perhaps in the
\textbf{References} section; see below). Please note that pages should be
numbered.

\subsubsection*{Acknowledgments}

Use unnumbered third level headings for the acknowledgments. All
acknowledgments go at the end of the paper. Do not include 
acknowledgments in the anonymized submission, only in the 
final paper. 

\subsubsection*{References}

\small{


[1] Vaswani, A., Shazeer, N., Parmar, N., Uszkoreit, J., Jones, L., 
Gomez, A. N., Kaiser, L., \& Polosukhin, I. (2017). Attention is All 
you Need. In {\it arXiv (Cornell University)} (Vol. 30, pp. 5998-6008). 
Cornell University. https://arxiv.org/pdf/1706.03762v5

[2] Cordonnier, J., Loukas, A., \& Jaggi, M. (2020b). On the 
relationship between selfattention and convolutional layers. In {\it International 
Conference on Learning Representations.} https://openreview.net/pdf?id=HJlnC1rKPB

[3] osovitskiy, A., Beyer, L., Kolesnikov, A. I., Weissenborn, D., Zhai, X., 
Unterthiner, T., Dehghani, M., Minderer, M., Heigold, G., Gelly, S., Uszkoreit, 
J., \& Houlsby, N. (2021). An Image is Worth 16x16 Words: Transformers for Image 
Recognition at Scale. In {\it International Conference on Learning Representations.}
https://openreview.net/pdf?id=YicbFdNTTy

}


\end{document}